\section{Introduction}
\label{sec:introduction}

This mini-challenge aims to generate high-quality (HQ) face images from corrupted low-quality (LQ) ones. The data used for this task is sourced from FFHQ. We provide a mini dataset with 5000 HQ images for training and 400 LQ-HQ image pairs for validation.

\subsection{Synthetic Data Generation}
\label{subsec:synthetic-data-generation}

We generate synthetic data following the methodology described in "Training Real-World Blind Super-Resolution with Pure Synthetic Data" by Wang et al. This process involves degrading HQ images into LQ images using various kernels, as shown in Figure~\ref{fig:kernels}. Each image has a unique kernel generated for it. The pipeline consists of the following steps:

\begin{enumerate}
    \item \textbf{Initialization}: Load configuration options and paths to ground-truth (GT) images. Initialize the file client based on the configuration.
    
    \item \textbf{Get Item}: Retrieve and read the GT image for a given index. Augment the GT image (horizontal flip, rotation).
    
    \item \textbf{Generate Kernels}: Select kernel size and type (sinc or mixed) based on probability. Generate and pad the kernel accordingly.
    
    \item \textbf{Apply Final Sinc Filter}: Optionally apply a final sinc filter.
    
    \item \textbf{Prepare Output}: Convert and format the image and kernels into tensors. Return tensors and the image path.
\end{enumerate}

\subsection{Evaluation}
\label{subsec:evaluation}

The final test set consists of 400 LQ images for evaluating model performance. We use the Peak Signal-to-Noise Ratio (PSNR) as the evaluation metric, which is a common measure in image processing to assess image quality. Higher PSNR values indicate better image quality, as shown in Equation~\ref{eq:psnr}:

\begin{equation}
    \text{PSNR} = 10 \cdot \log_{10} \left( \frac{\text{MAX}_I^2}{\text{MSE}} \right)
    \label{eq:psnr}
\end{equation}
