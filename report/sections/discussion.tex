\section{Discussion}
\label{sec:Discussion} % Label for referencing

This project brings to light the delicate balance required in super-resolution between optimizing for human visual preferences and computational metrics. While human observers prioritize perceptual and semantic details, computational models tend to focus on precise pixel-based evaluations.

The use of Generative Adversarial Networks (GANs) equipped with perceptual loss functions is a promising approach to reconcile these perspectives. Perceptual loss allows for the generation of images that are more visually pleasing to humans by mimicking the way human vision processes images. However, this often results in lower Peak Signal-to-Noise Ratio (PSNR) scores, as the focus shifts from exact pixel accuracy to more qualitative aspects of image quality.

Despite the potential benefits of perceptual loss, it was not utilized in this experiment due to constraints on using external data. Understanding that the highest PSNR scores do not always correlate with the most visually appealing images is crucial. 
